% Options for packages loaded elsewhere
\PassOptionsToPackage{unicode}{hyperref}
\PassOptionsToPackage{hyphens}{url}
\PassOptionsToPackage{dvipsnames,svgnames,x11names}{xcolor}
%
\documentclass[
  a4paper,
]{scrreport}

\usepackage{amsmath,amssymb}
\usepackage{iftex}
\ifPDFTeX
  \usepackage[T1]{fontenc}
  \usepackage[utf8]{inputenc}
  \usepackage{textcomp} % provide euro and other symbols
\else % if luatex or xetex
  \usepackage{unicode-math}
  \defaultfontfeatures{Scale=MatchLowercase}
  \defaultfontfeatures[\rmfamily]{Ligatures=TeX,Scale=1}
\fi
\usepackage{lmodern}
\ifPDFTeX\else  
    % xetex/luatex font selection
\fi
% Use upquote if available, for straight quotes in verbatim environments
\IfFileExists{upquote.sty}{\usepackage{upquote}}{}
\IfFileExists{microtype.sty}{% use microtype if available
  \usepackage[]{microtype}
  \UseMicrotypeSet[protrusion]{basicmath} % disable protrusion for tt fonts
}{}
\makeatletter
\@ifundefined{KOMAClassName}{% if non-KOMA class
  \IfFileExists{parskip.sty}{%
    \usepackage{parskip}
  }{% else
    \setlength{\parindent}{0pt}
    \setlength{\parskip}{6pt plus 2pt minus 1pt}}
}{% if KOMA class
  \KOMAoptions{parskip=half}}
\makeatother
\usepackage{xcolor}
\setlength{\emergencystretch}{3em} % prevent overfull lines
\setcounter{secnumdepth}{5}
% Make \paragraph and \subparagraph free-standing
\ifx\paragraph\undefined\else
  \let\oldparagraph\paragraph
  \renewcommand{\paragraph}[1]{\oldparagraph{#1}\mbox{}}
\fi
\ifx\subparagraph\undefined\else
  \let\oldsubparagraph\subparagraph
  \renewcommand{\subparagraph}[1]{\oldsubparagraph{#1}\mbox{}}
\fi


\providecommand{\tightlist}{%
  \setlength{\itemsep}{0pt}\setlength{\parskip}{0pt}}\usepackage{longtable,booktabs,array}
\usepackage{calc} % for calculating minipage widths
% Correct order of tables after \paragraph or \subparagraph
\usepackage{etoolbox}
\makeatletter
\patchcmd\longtable{\par}{\if@noskipsec\mbox{}\fi\par}{}{}
\makeatother
% Allow footnotes in longtable head/foot
\IfFileExists{footnotehyper.sty}{\usepackage{footnotehyper}}{\usepackage{footnote}}
\makesavenoteenv{longtable}
\usepackage{graphicx}
\makeatletter
\def\maxwidth{\ifdim\Gin@nat@width>\linewidth\linewidth\else\Gin@nat@width\fi}
\def\maxheight{\ifdim\Gin@nat@height>\textheight\textheight\else\Gin@nat@height\fi}
\makeatother
% Scale images if necessary, so that they will not overflow the page
% margins by default, and it is still possible to overwrite the defaults
% using explicit options in \includegraphics[width, height, ...]{}
\setkeys{Gin}{width=\maxwidth,height=\maxheight,keepaspectratio}
% Set default figure placement to htbp
\makeatletter
\def\fps@figure{htbp}
\makeatother

\newcommand{\NN}{\mathbb{N}}
\newcommand{\ZZ}{\mathbb{Z}}
\newcommand{\QQ}{\mathbb{Q}}
\newcommand{\RR}{\mathbb{R}}
\newcommand{\CC}{\mathbb{C}}
\makeatletter
\@ifpackageloaded{bookmark}{}{\usepackage{bookmark}}
\makeatother
\makeatletter
\@ifpackageloaded{caption}{}{\usepackage{caption}}
\AtBeginDocument{%
\ifdefined\contentsname
  \renewcommand*\contentsname{Tabla de contenidos}
\else
  \newcommand\contentsname{Tabla de contenidos}
\fi
\ifdefined\listfigurename
  \renewcommand*\listfigurename{Listado de Figuras}
\else
  \newcommand\listfigurename{Listado de Figuras}
\fi
\ifdefined\listtablename
  \renewcommand*\listtablename{Listado de Tablas}
\else
  \newcommand\listtablename{Listado de Tablas}
\fi
\ifdefined\figurename
  \renewcommand*\figurename{Figura}
\else
  \newcommand\figurename{Figura}
\fi
\ifdefined\tablename
  \renewcommand*\tablename{Tabla}
\else
  \newcommand\tablename{Tabla}
\fi
}
\@ifpackageloaded{float}{}{\usepackage{float}}
\floatstyle{ruled}
\@ifundefined{c@chapter}{\newfloat{codelisting}{h}{lop}}{\newfloat{codelisting}{h}{lop}[chapter]}
\floatname{codelisting}{Listado}
\newcommand*\listoflistings{\listof{codelisting}{Listado de Listados}}
\usepackage{amsthm}
\theoremstyle{definition}
\newtheorem{exercise}{Ejercicio}[chapter]
\theoremstyle{remark}
\AtBeginDocument{\renewcommand*{\proofname}{Prueba}}
\newtheorem*{remark}{Observación}
\newtheorem*{solution}{Solución}
\newtheorem{refremark}{Observación}[chapter]
\newtheorem{refsolution}{Solución}[chapter]
\makeatother
\makeatletter
\makeatother
\makeatletter
\@ifpackageloaded{caption}{}{\usepackage{caption}}
\@ifpackageloaded{subcaption}{}{\usepackage{subcaption}}
\makeatother
\ifLuaTeX
\usepackage[bidi=basic]{babel}
\else
\usepackage[bidi=default]{babel}
\fi
\babelprovide[main,import]{spanish}
% get rid of language-specific shorthands (see #6817):
\let\LanguageShortHands\languageshorthands
\def\languageshorthands#1{}
\ifLuaTeX
  \usepackage{selnolig}  % disable illegal ligatures
\fi
\usepackage{bookmark}

\IfFileExists{xurl.sty}{\usepackage{xurl}}{} % add URL line breaks if available
\urlstyle{same} % disable monospaced font for URLs
\hypersetup{
  pdftitle={Test de nivel de Cálculo},
  pdfauthor={Alfredo Sánchez Alberca},
  pdflang={es},
  colorlinks=true,
  linkcolor={blue},
  filecolor={Maroon},
  citecolor={Blue},
  urlcolor={Blue},
  pdfcreator={LaTeX via pandoc}}

\title{Test de nivel de Cálculo}
\author{Alfredo Sánchez Alberca}
\date{2023-01-06}

\begin{document}
\begin{titlepage}

%\AddToShipoutPicture*{\put(0,0){\includegraphics[scale=0.8]{img/background2}}} % Imagen de fondo, requiere el paquete eso-pic.
\begin{center}
\vspace*{5cm}

\Huge
{\textbf{\textsf{Test de nivel de Cálculo}}}

\vspace{0.5cm}
\LARGE
{\textbf{\textsf{}}}

\vspace{1.5cm}

\includegraphics[width=0.4\textwidth]{img/logos/sticker.png}
\end{center}

\vfill

\begin{flushleft}
\begin{tabular}{ll}
\includegraphics[width=0.1\textwidth]{img/logos/aprendeconalf.png} & \parbox[b]{5cm}{\Large\textsf{Alfredo
Sánchez
Alberca}\\ \textsf{asalber@ceu.es} \\ \textsf{https://aprendeconalf.es}}
\end{tabular}
\end{flushleft}
\end{titlepage}
\renewcommand*\contentsname{Tabla de contenidos}
{
\hypersetup{linkcolor=}
\setcounter{tocdepth}{2}
\tableofcontents
}
\bookmarksetup{startatroot}

\chapter*{Prefacio}\label{prefacio}
\addcontentsline{toc}{chapter}{Prefacio}

\markboth{Prefacio}{Prefacio}

¡Bienvenido al test de nivel de Cálculo!

Este test está pensado para determinar el nivel de Cálculo de los
alumnos que comienzan un grado de Ciencias o Ingeniería y cuáles son los
temas con más carencias que debería reforzar.

\section*{Licencia}\label{licencia}
\addcontentsline{toc}{section}{Licencia}

\markright{Licencia}

Esta obra está bajo una licencia Reconocimiento -- No comercial --
Compartir bajo la misma licencia 3.0 España de Creative Commons. Para
ver una copia de esta licencia, visite
\url{https://creativecommons.org/licenses/by-nc-sa/3.0/es/}.

Con esta licencia eres libre de:

\begin{itemize}
\tightlist
\item
  Copiar, distribuir y mostrar este trabajo.
\item
  Realizar modificaciones de este trabajo.
\end{itemize}

Bajo las siguientes condiciones:

\begin{itemize}
\item
  \textbf{Reconocimiento}. Debe reconocer los créditos de la obra de la
  manera especificada por el autor o el licenciador (pero no de una
  manera que sugiera que tiene su apoyo o apoyan el uso que hace de su
  obra).
\item
  \textbf{No comercial}. No puede utilizar esta obra para fines
  comerciales.
\item
  \textbf{Compartir bajo la misma licencia}. Si altera o transforma esta
  obra, o genera una obra derivada, sólo puede distribuir la obra
  generada bajo una licencia idéntica a ésta.
\end{itemize}

Al reutilizar o distribuir la obra, tiene que dejar bien claro los
términos de la licencia de esta obra.

Estas condiciones pueden no aplicarse si se obtiene el permiso del
titular de los derechos de autor.

Nada en esta licencia menoscaba o restringe los derechos morales del
autor.

\bookmarksetup{startatroot}

\chapter{Test de nivel de Cálculo}\label{test-de-nivel-de-cuxe1lculo}

\begin{exercise}[]\protect\hypertarget{exr-01}{}\label{exr-01}

Si una máquina fabrica 10800 pastillas en una hora, ¿cuántas pastillas
fabrica la máquina por segundo?

\begin{enumerate}
\def\labelenumi{\alph{enumi}.}
\item
  \(1\).
\item
  \(180\).
\item
  \(3\). (*)
\item
  \(5\).
\item
  Las otras opciones son falsas.
\end{enumerate}

\end{exercise}

\begin{exercise}[]\protect\hypertarget{exr-02}{}\label{exr-02}

El resultado de la operación \(\dfrac{\sqrt{81}}{\sqrt{9}}\) es

\begin{enumerate}
\def\labelenumi{\alph{enumi}.}
\item
  \(\sqrt{3}\).
\item
  \(\sqrt{9}\). (*)
\item
  \(\sqrt{27}\).
\item
  \(\sqrt{7}\).
\item
  Las otras opciones son falsas.
\end{enumerate}

\end{exercise}

\begin{exercise}[]\protect\hypertarget{exr-03}{}\label{exr-03}

La solución de la ecuación \(5^x=625\) es

\begin{enumerate}
\def\labelenumi{\alph{enumi}.}
\item
  \(5\).
\item
  \(4\). (*)
\item
  \(3\).
\item
  \(6\).
\item
  Las otras opciones son falsas.
\end{enumerate}

\end{exercise}

\begin{exercise}[]\protect\hypertarget{exr-04}{}\label{exr-04}

El resultado de la operación \((x^2+3x+2)(x+2)\) es

\begin{enumerate}
\def\labelenumi{\alph{enumi}.}
\item
  \(x^3+5x^2+7x+4\).
\item
  \(x^3+2x^2+7x+2\).
\item
  \(x^3+2x^2+5x+4\).
\item
  \(x^3+2x^2+8x+4\).
\item
  Las otras opciones son falsas. (*)
\end{enumerate}

\end{exercise}

\begin{exercise}[]\protect\hypertarget{exr-05}{}\label{exr-05}

Si dividimos \(x^5-32\) entre \(x-2\) se obtiene un cociente \(C(x)\) y
un resto \(R(x)\) igual a:

\begin{enumerate}
\def\labelenumi{\alph{enumi}.}
\item
  \(C(x) = x^4-2x^3-4x^2-8x-16\) y \(R(x)=0\).
\item
  \(C(x) = x^4+2x^3+4x^2+8x+16\) y \(R(x)=0\). (*)
\item
  \(C(x)=x^4+2x^3+4x^2+8x+16\) y \(R(x)=2\).
\item
  \(C(x)= x^4-2x^3-4x^2-8x-16\) y \(R(x)=2\).
\item
  Las otras opciones son falsas.
\end{enumerate}

\end{exercise}

\begin{exercise}[]\protect\hypertarget{exr-06}{}\label{exr-06}

La expresión \(\dfrac{a}{b+a}-\dfrac{b}{b-a}\), siendo \(a\) y \(b\) dos
constantes, es igual a:

\begin{enumerate}
\def\labelenumi{\alph{enumi}.}
\item
  \(-1\).
\item
  \(\dfrac{a^2+b^2}{a^2-b^2}\). (*)
\item
  \(\dfrac{a^2+b^2}{b^2-a^2}\).
\item
  \(a+b\).
\item
  Las otras opciones son falsas.
\end{enumerate}

\end{exercise}

\begin{exercise}[]\protect\hypertarget{exr-07}{}\label{exr-07}

Si \(x\) e \(y\) están relacionadas mediante la ecuación \(3x+y=2\)
entonces

\begin{enumerate}
\def\labelenumi{\alph{enumi}.}
\item
  \(y\) aumenta una unidad por cada unidad que aumenta \(x\).
\item
  \(y\) disminuye 2 unidades por cada unidad que aumenta \(x\).
\item
  \(x\) aumenta un tercio de unidad por cada unidad que aumenta \(y\).
\item
  \(y\) aumenta 3 unidades por cada unidad que aumenta \(x\).
\item
  Las otras opciones son falsas. (*)
\end{enumerate}

\end{exercise}

\begin{exercise}[]\protect\hypertarget{exr-08}{}\label{exr-08}

Las raíces del polinomio \(x^2-5x+6\) son

\begin{enumerate}
\def\labelenumi{\alph{enumi}.}
\item
  \(-3\) y \(-2\).
\item
  \(2\) y \(3\). (*)
\item
  \(-2\) y \(3\).
\item
  No tiene raíces reales.
\item
  Las otras opciones son falsas.
\end{enumerate}

\end{exercise}

\begin{exercise}[]\protect\hypertarget{exr-9}{}\label{exr-9}

La distancia entre los puntos del plano \((2,3)\) y \((5,7)\) es

\begin{enumerate}
\def\labelenumi{\alph{enumi}.}
\item
  \(4\).
\item
  \(5\). (*)
\item
  \(3\).
\item
  \(6\).
\item
  Las otras opciones son falsas.
\end{enumerate}

\end{exercise}

\begin{exercise}[]\protect\hypertarget{exr-10}{}\label{exr-10}

El producto escalar de los vectores \((1,0,-2)\) y \((3,-1,2)\) es

\begin{enumerate}
\def\labelenumi{\alph{enumi}.}
\item
  \(-1\). (*)
\item
  \(0\).
\item
  \(-2\).
\item
  \(3\).
\item
  Las otras opciones son falsas.
\end{enumerate}

\end{exercise}

\begin{exercise}[]\protect\hypertarget{exr-11}{}\label{exr-11}

Dada la matriz

\[
A=
\left(
\begin{array}{cc}
1 & 2 \\
-1 & 1 \\
\end{array}
\right)
\]

su determinante vale

\begin{enumerate}
\def\labelenumi{\alph{enumi}.}
\item
  \(0\).
\item
  \(-3\).
\item
  \(3\). (*)
\item
  \(2\).
\item
  Las otras opciones son falsas.
\end{enumerate}

\end{exercise}

\begin{exercise}[]\protect\hypertarget{exr-12}{}\label{exr-12}

Dado el sistema de ecuaciones

\[
\left\{
\begin{array}{l}
2x-y=-1 \\
kx+y=2 \\
\end{array}
\right.
\]

¿Cuál de las siguientes afirmaciones es correcta?

\begin{enumerate}
\def\labelenumi{\alph{enumi}.}
\item
  Para ningún valor de \(k\) el sistema es compatible.
\item
  Para sólo un valor de \(k\) el sistema es compatible.
\item
  El sistema es homogéneo.
\item
  Si el sistema es compatible, tiene que ser determinado. (*)
\item
  Las otras afirmaciones son falsas.
\end{enumerate}

\end{exercise}

\begin{exercise}[]\protect\hypertarget{exr-13}{}\label{exr-13}

La expresión \(2\ln(a)-3\ln(b)\), siendo \(a\) y \(b\) dos constantes,
es igual a:

\begin{enumerate}
\def\labelenumi{\alph{enumi}.}
\item
  \(\ln(a^2-b^3)\).
\item
  \(\ln(a^2b^3)\).
\item
  \(\ln\left(\frac{a^2}{b^3}\right)\). (*)
\item
  \(e^{2a+3b}\).
\item
  Las otras opciones son falsas.
\end{enumerate}

\end{exercise}

\begin{exercise}[]\protect\hypertarget{exr-14}{}\label{exr-14}

Dada la función \(a^x\), con \(a>0\) una constante, ¿cuál de las
siguientes afirmaciones es falsa?

\begin{enumerate}
\def\labelenumi{\alph{enumi}.}
\item
  Tiene como dominio todo \(\mathbb{R}\).
\item
  No puede tomar valores negativos.
\item
  Es creciente en todo su dominio. (*)
\item
  No tiene extremos relativos.
\item
  Su gráfica pasa por el punto \((0,1)\).
\end{enumerate}

\end{exercise}

\begin{exercise}[]\protect\hypertarget{exr-15}{}\label{exr-15}

La función \(g(x)= 2\cos(x/2)\)

\begin{enumerate}
\def\labelenumi{\alph{enumi}.}
\item
  Tiene periodo \(2\pi\).
\item
  Tiene periodo \(4\pi\). (*)
\item
  Tiene periodo \(\pi\).
\item
  Tiene periodo \(\pi/2\).
\item
  Las otras opciones son falsas.
\end{enumerate}

\end{exercise}

\begin{exercise}[]\protect\hypertarget{exr-16}{}\label{exr-16}

La función inversa de la función \(f(x)=\operatorname{arcsen}(x^3)\) es

\begin{enumerate}
\def\labelenumi{\alph{enumi}.}
\item
  \(g(y)=\operatorname{sen}\left(\frac{1}{y^3}\right)\).
\item
  \(g(y)=\operatorname{sen}(\sqrt[3]{y})\).
\item
  \(g(y)=\frac{1}{\operatorname{arcsen}(y^3)}\)
\item
  \(g(y)=\sqrt[3]{\operatorname{sen}(y)}\). (*)
\item
  Las otras opciones son falsas.
\end{enumerate}

\end{exercise}

\begin{exercise}[]\protect\hypertarget{exr-17}{}\label{exr-17}

El límite \(\lim_{x\to\infty} \dfrac{2x^2-5}{\sqrt{3x^4-2}}\) da como
resultado

\begin{enumerate}
\def\labelenumi{\alph{enumi}.}
\item
  \(0\).
\item
  \(\frac{2}{3}\).
\item
  \(\infty\).
\item
  \(\frac{2\sqrt{3}}{3}\). (*)
\item
  Las otras opciones son falsas.
\end{enumerate}

\end{exercise}

\begin{exercise}[]\protect\hypertarget{exr-18}{}\label{exr-18}

El límite \(\lim_{x\to 0}\dfrac{\operatorname{sen}(x)}{x}\) vale

\begin{enumerate}
\def\labelenumi{\alph{enumi}.}
\item
  \(0\).
\item
  \(\infty\).
\item
  \(1\). (*)
\item
  No existe.
\item
  Las otras opciones son falsas.
\end{enumerate}

\end{exercise}

\begin{exercise}[]\protect\hypertarget{exr-19}{}\label{exr-19}

La función \(h(x)=\dfrac{3x^2+2x+1}{4x^2+5}\)

\begin{enumerate}
\def\labelenumi{\alph{enumi}.}
\item
  Tiene una asíntota horizontal \(y=\frac{3}{4}\). (*)
\item
  Tiene una asíntota vertical \(x=\frac{\sqrt{5}}{2}\).
\item
  Tiene una asíntota oblicua \(y=\frac{4}{3}x-1\).
\item
  No tiene asíntotas.
\item
  Las otras opciones son falsas.
\end{enumerate}

\end{exercise}

\begin{exercise}[]\protect\hypertarget{exr-20}{}\label{exr-20}

La función \(f(x)=\frac{x^3-x}{x+1}\)

\begin{enumerate}
\def\labelenumi{\alph{enumi}.}
\item
  Tiene una discontinuidad evitable en \(x=1\). (*)
\item
  Tiene una discontinuidad de salto finito en \(x=1\).
\item
  Tiene una discontinuidad de salto infinito en \(x=1\).
\item
  Es continua en \(x=1\).
\item
  Las otras opciones son falsas.
\end{enumerate}

\end{exercise}

\begin{exercise}[]\protect\hypertarget{exr-21}{}\label{exr-21}

La derivada de la función \(f(x)=\ln(\sqrt{x/2})\) es

\begin{enumerate}
\def\labelenumi{\alph{enumi}.}
\item
  \(\frac{1}{\sqrt{x/2}}\).
\item
  \(\frac{1}{x/2}\).
\item
  \(\frac{1}{2x}\). (*)
\item
  \(\frac{1}{x}\).
\item
  Las otras opciones son falsas.
\end{enumerate}

\end{exercise}

\begin{exercise}[]\protect\hypertarget{exr-22}{}\label{exr-22}

La recta tangente a la gráfica de la función \(f(x)=x^3-2x^2+x-2\) en
\(x=2\) vale

\begin{enumerate}
\def\labelenumi{\alph{enumi}.}
\item
  \(y=5x\).
\item
  \(y=2x-5\).
\item
  \(y=5x-10\). (*)
\item
  \(y=2x+10\).
\item
  Las otras opciones son falsas.
\end{enumerate}

\end{exercise}

\begin{exercise}[]\protect\hypertarget{exr-23}{}\label{exr-23}

La función \(h(x)=x^2+2x+1\),

\begin{enumerate}
\def\labelenumi{\alph{enumi}.}
\item
  Tiene un máximo relativo en \(x=-1\).
\item
  Tiene un mínimo relativo en \(x=1\).
\item
  Tiene un punto de inflexión en \(x=1\).
\item
  Tiene un punto de inflexión en \(x=-1\).
\item
  Las otras opciones son falsas. (*)
\end{enumerate}

\end{exercise}

\begin{exercise}[]\protect\hypertarget{exr-24}{}\label{exr-24}

La integral \(\int x \ln(x) \, dx\) es

\begin{itemize}
\tightlist
\item
  \(\frac{x^2 \ln(x)}{2} - \frac{x^2}{4} + C\)
\item
  \(\frac{x^2 \ln(x)}{2} + C\)
\item
  \(x \ln(x) - x + C\)
\item
  \(\frac{x^2 \ln(x)}{2} - x + C\)
\item
  Las otras opciones son falsas. (*)
\end{itemize}

\end{exercise}

\begin{exercise}[]\protect\hypertarget{exr-25}{}\label{exr-25}

El área encerrada entre la gráfica de la función \(g(x)=\cos(x/2)\) y el
eje \(x\) en el intervalo \([0,2\pi]\) es

\begin{enumerate}
\def\labelenumi{\alph{enumi}.}
\item
  \(0\).
\item
  \(1\).
\item
  \(\pi\).
\item
  \(4\). (*)
\item
  Las otras opciones son falsas.
\end{enumerate}

\end{exercise}



\end{document}
